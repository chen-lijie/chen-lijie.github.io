\documentclass[12pt]{article}
\usepackage[margin=1in]{geometry}
\usepackage{amsmath, amsfonts, amssymb, amsthm}
\usepackage{mathtools}
\usepackage{enumerate}
\usepackage{fancyhdr}
\usepackage{graphicx}
\usepackage{hyperref}
\usepackage{xspace}

% Theorem environments
\newtheorem{theorem}{Theorem}
\newtheorem{lemma}[theorem]{Lemma}
\newtheorem{proposition}[theorem]{Proposition}
\newtheorem{corollary}[theorem]{Corollary}

% Some complexity-class macros (adapted from HW3)
\newcommand{\NP}{\text{NP}}
\newcommand{\NEXP}{\text{NEXP}}
\newcommand{\coNEXP}{\text{coNEXP}}
\newcommand{\ZPP}{\text{ZPP}}
\newcommand{\TCone}{\text{TC}^1}
\newcommand{\poly}{\mathrm{poly}}
\newcommand{\eps}{\varepsilon}
\newcommand{\oracle}{\mathcal{O}}
\newcommand{\ACzero}{\text{AC}^0}
\newcommand{\TCzero}{\text{TC}^0}
\newcommand{\Ppoly}{\text{P}/\poly}

% Header setup
\pagestyle{fancy}
\fancyhf{}
\setlength{\headheight}{15pt}
\lhead{CS 278: Computational Complexity Theory}
\rhead{Fall 2025}
\lfoot{Name: \underline{\hspace{3cm}}}
\rfoot{Page \thepage}

\title{CS 278: Computational Complexity Theory\\
       Homework 4}
\author{Due: \textbf{Sunday, December 14, 2025, 11:59pm Pacific Time}}
\date{Fall 2025}

\begin{document}

\maketitle

\begin{center}
\textbf{Instructions:}
\end{center}

\begin{itemize}
    \item Collaboration is allowed, but you must write up your solutions \emph{by yourself} and in your own words.  List all collaborators and any external resources you used.
    \item Write your solutions in \LaTeX\xspace and submit a single PDF to Gradescope under ``HW4''.
    \item \textbf{Deadline:} 11:59pm Pacific Time on \textbf{Sunday, December 14, 2025}.
    \item Late submissions lose \textbf{10\%} per day (e.g., three days late $\rightarrow 0.9^3$ of your score).
    \item The maximum \emph{raw} score of this homework is $160$. There are 2 problems, each worth $80$ points, with many subparts. However, you only need $100$ points to get full credit.
    \item Let $n = \min\{\text{your raw score on this homework}, 100\}$. The contribution of this homework to your course grade is
    \[
        a_4 = \frac{n}{100} \cdot 12.5.
    \]
    \item Let $a_1,a_2,a_3,a_4$ be the contributions from Homeworks 1--4. Your final homework component is
    \[
        \min(a_1 + a_2 + a_3 + a_4, 50).
    \]
    \item In other words, you do \emph{not} need to solve every problem on every homework to get full homework credit.
\end{itemize}

\newpage

%%%%%%%%%%%%%%%%%%%%%%%%%%%%%%%%%%%%%%%%%%%%%%%%%%%%%%%%%%%%
\section{Problem 1 (80 pts): Near-linear MA protocol for Counting Orthogonal Vectors}

The \emph{Counting Orthogonal Vectors (\#OV)} problem is defined as follows: given two sets of vectors $A = \{a_1, \dots, a_n\}$ and $B = \{b_1, \dots, b_n\}$ where each $a_i, b_i \in \{0,1\}^d$, count the number of pairs $(i,j)$ such that $\langle a_i, b_j \rangle = 0$. Here the inner product is over the integers (or reals), i.e., $\sum_{k=1}^d a_{i,k} b_{j,k} = 0$.
The ``standard'' algorithm takes $O(n^2 d)$ time. In this problem, we will show a near-linear-time MA protocol for \#OV. Let $p$ be a prime number such that $p > n$.

\begin{enumerate}[(a)]
    \item \textbf{(20 pts) Polynomial Formulation.}
    \\
    Let $\mathbb{F}_p$ be a prime field with $p > n$. 
    For a fixed vector $v \in \{0,1\}^d$, define a multivariate polynomial $P_v(x_1, \dots, x_d)$ over $\mathbb{F}_p$ such that for any vector $u \in \{0,1\}^d$:
    \[
        P_v(u) = 1 \iff \langle u, v \rangle = 0
    \]
    and
    \[
        P_v(u) = 0 \iff \langle u, v \rangle \neq 0.
    \]
    Ideally, the degree of $P_v$ should be relatively low (e.g., degree $d$ or close to it).

    \textbf{Hint:} write $\langle u, v \rangle$ as a simple $\textsf{AC}^0$ circuit and arithmetize it.

    \item \textbf{(15 pts) Batch Verification via Polynomials.}
    
    Building on the ideas from part (a), for the given set $A = \{a_1, \dots, a_n\}$, define a polynomial $P_A(x_1,\dots,x_d)$ such that for any vector $u \in \{0,1\}^d$: we have $P_A(u)$ is the number of vectors $a_i \in A$ such that $\langle a_i, u \rangle = 0$.

    Furthemore, show that $P_A(x_1, \dotsc, x_d)$ can be computed in time $\tilde{O}(n \cdot \poly(d))$.
    
    \item \textbf{(20 pts) The Protocol.}
    
    Show that you can construct another univariate polynomial $Q_A(x)$ such that $Q_A(i)$ equals $P_A(b_i)$ for all $i \in [n]$, and $Q_A(x)$ can be computed in $\tilde{O}(n \cdot \poly(d))$ time.

    \paragraph{Hint:}You can use the fact that given a univariate polynomial $P(x)$ of degree $T$ and $n$ inputs $x_1, \dots, x_n$, you can compute $P(x_1), \dots, P(x_n)$ in time $\tilde{O}(n + T)$.

    Show that given the correct $Q_A(u)$ you can solve the \#OV problem in time $\tilde{O}(n \cdot \poly(d))$.
    
    \item \textbf{(25 pts) Soundness and Completeness.}
    
    Design an MA protocol such that if Merlin sends Arthur the correct $Q_A(u)$, then Arthur accepts with probability 1; and otherwise Arthur rejects with probability at least $1-1/\poly(n)$. Show how this protocol can be used to solve the \#OV problem in time $\tilde{O}(n \cdot \poly(d))$.

    \textbf{Hint:} think about the IP=PSPACE protocol covered in class.
\end{enumerate}

\newpage

%%%%%%%%%%%%%%%%%%%%%%%%%%%%%%%%%%%%%%%%%%%%%%%%%%%%%%%%%%%%
\section{Problem 2 (80 pts): Complexity of Transformers}

This problem asks you to reason about very simple complexity-theoretic idealizations of Transformer-style sequence models and Chain-of-Thought (CoT) reasoning.  You do \emph{not} need to know anything about practical Transformers beyond what is stated here.

\paragraph{Basic notions.}
\begin{itemize}
    \item A \textbf{token} is just a symbol from some finite alphabet $\Sigma$ (you can think of $\Sigma = \{0,1\}$ in this problem).
    \item An \textbf{autoregressive (AR) generative model} for length-$T$ outputs is a probabilistic algorithm $G_{\mathrm{AR}}$ which, on input $x$ and random coins $r$, produces tokens $y_1,\dots,y_T$ one-by-one.  Formally, in round $t$ it outputs $y_t$ as a (randomized) function of $(x,y_1,\dots,y_{t-1},r)$, and the total running time over all $T$ rounds is $\poly(|x|,T)$.
    \item $\ACzero$ is the class of constant-depth, polynomial-size Boolean circuits with unbounded fan-in AND, OR, and NOT gates.  $\TCzero$ is defined similarly, but also allows unbounded fan-in \emph{majority} (threshold) gates.
    \item A \textbf{one-way function (OWF)} is a function $f:\{0,1\}^n \to \{0,1\}^n$ that is computable in time $\poly(n)$ but is hard to invert on a random input: let $x \in \{0,1\}^n$ be a random input, and $y = f(x)$. Then for any $\poly(n)$-time adversary $A$, the probability that $A(y) = x'$ such that $f(x') = y$ is negligible.
    
    \item A \textbf{Chain-of-Thought (CoT) decoder} is an AR model that, on input $x$, first emits a sequence of intermediate ``reasoning'' tokens $z_1,z_2,\dots,z_L$ (the \emph{chain-of-thought}), and then emits a final answer token $y$.  ``Thinking in dots'' refers to the special case where all the $z_i$ are the same filler token ``$\cdot$''.
\end{itemize}

\bigskip

\noindent In all parts below, high-level arguments and sketches are fine; you do \emph{not} need to give fully formal reductions, but you should clearly indicate what assumptions you are using and why they are relevant.

\begin{enumerate}[(a)]
    \item \textbf{(20 pts) Limitations of AR models under OWF.}
    Assume one-way functions exist. Prove that there exists a distribution $D$ over $\{0,1\}^n$ that cannot be generated by any polynomial-time AR model, but can be generated by a polynomial-time algorithm that is not AR.

    \item \textbf{(25 pts) ``Thinking in dots'' with a $\TCzero$ decoder.}
    Assume that a constant-depth decoder is in $\TCzero$, show that the thinking-in-dots decoder is in $\TCzero$. Here, the thinking-in-dots decoder works by in every decoding step it either emits a filler token ``$\cdot$'' or the answer token and halts, and it is guaranteed that it thinks in dots for at most $\poly(n)$ steps.

    \item \textbf{(35 pts) Why CoT does not give $\ACzero$ decoders Parity.}
    \\
    Now suppose each decoding step (including the final answer step) is computed by a uniform $\ACzero$ circuit on the current state.
    As above, on input $x \in \{0,1\}^n$, the decoder first produces a chain-of-thought $z_1,\dots,z_L$ of length $L = n^{0.99}$, then a final answer $y$.
    Assume the decoding is deterministic (e.g., greedy decoding).

    You may use the following standard average-case hardness theorem for Parity:
    \begin{quote}
        \emph{Theorem (Parity vs.\ $\ACzero$, informal).}
        For every constant depth $d$, for every $\delta \in (0,1)$, it holds for all sufficiently large $n$, every depth-$d$ $\ACzero$ circuit $C$ of polynomial size satisfies
        \[
            \Pr_{x \leftarrow \{0,1\}^n}[C(x) = \mathrm{Parity}(x)] \le \tfrac{1}{2} + 2^{-\Omega(n^\delta)}.
        \]
    \end{quote}

    Give a careful but high-level argument that even with a chain-of-thought of length $L = n^{0.99}$, such an $\ACzero$-based decoder still cannot compute the Parity function with high success probability on a random input $x$.
\end{enumerate}

\vfill
\noindent\textbf{(Optional) Solution placeholders}\\
You may use the following headings for your writeup; remove them if not needed.

\newpage
\section*{Solution to Problem 1}

\subsection*{Part (a): Polynomial Formulation}

\subsection*{Part (b): Batch Verification via Univariate Polynomials}

\subsection*{Part (c): The Protocol}

\subsection*{Part (d): Soundness and Completeness}

\newpage
\section*{Solution to Problem 2}

\subsection*{Part (a): Limitations of AR models under OWF}

\subsection*{Part (b): ``Thinking in dots'' with a $\TCzero$ decoder}

\subsection*{Part (c): Why CoT does not give $\ACzero$ decoders Parity}

\end{document}
