
\documentclass[12pt]{article}
\usepackage[margin=1in]{geometry}
\usepackage{amsmath, amsfonts, amssymb, amsthm}
\usepackage{mathtools}
\usepackage{enumerate}
\usepackage{fancyhdr}
\usepackage{graphicx}
\usepackage{hyperref}
\usepackage{xspace}

% Theorem environments
\newtheorem{theorem}{Theorem}
\newtheorem{lemma}[theorem]{Lemma}
\newtheorem{proposition}[theorem]{Proposition}
\newtheorem{corollary}[theorem]{Corollary}
\newtheorem{definition}[theorem]{Definition}
\newtheorem{claim}[theorem]{Claim}
\newtheorem{remark}[theorem]{Remark}
\newtheorem{hint}{Hint}

% Common complexity classes / notation
\newcommand{\DTIME}{\text{DTIME}}
\newcommand{\NTIME}{\text{NTIME}}
\newcommand{\TIME}{\text{TIME}}
\newcommand{\SPACE}{\text{SPACE}}
\newcommand{\NSPACE}{\text{NSPACE}}
\newcommand{\SIZE}{\text{SIZE}}
\newcommand{\NP}{\text{NP}}
\newcommand{\NEXP}{\text{NEXP}}
\newcommand{\coNEXP}{\text{coNEXP}}
\newcommand{\ZPP}{\text{ZPP}}
\newcommand{\TCone}{\text{TC}^1}
\newcommand{\poly}{\mathrm{poly}}
\newcommand{\eps}{\varepsilon}
\newcommand{\oracle}{\mathcal{O}}

% Header setup
\pagestyle{fancy}
\fancyhf{}
\lhead{CS 278: Computational Complexity Theory}
\rhead{Fall 2025}
\lfoot{Name: \underline{\hspace{3cm}}}
\rfoot{Page \thepage}

\title{CS 278: Computational Complexity Theory\\
       Homework 2}
\author{Due: \textbf{October 24th, 2025}}
\date{Fall 2025}

\begin{document}

\maketitle

\begin{center}
\textbf{Instructions:}

\begin{itemize}
    \item Collaboration is allowed but solutions must be written independently. List collaborators and external resources.
    \item Write your solutions in \LaTeX\xspace and submit a single PDF by email to \texttt{lijiechen@berkeley.edu} with subject ``CS 278 - Homework 2 -- [Your Name]''.
    \item Name your file \texttt{CS278-HW2-[YourName].pdf}. \textbf{Deadline:} 11:59pm Pacific Time October 24th, 2025.
    \item Late submissions lose \textbf{10\%} per day (e.g., three days late $\rightarrow 0.9^3$ of your score).
    \item This homework has 4 problems, each with multiple parts, totaling \textbf{160 points}. As in HW1, our course policy on homework aggregation applies unchanged.
\end{itemize}
\end{center}

\vspace{0.5cm}

\newpage

\section{Problem 1 (40 pts): A circuit lower bound for quadratic space}

Show that languages decidable in quadratic space do not admit linear-size circuits.

\paragraph{Statement.}
Prove that $\SPACE[n^2] \not\subseteq \SIZE(O(n))$. In other words, there is a language $L \in \SPACE[n^2]$ such that every Boolean circuit family computing $L$ on inputs of length $n$ has size $\omega(n)$ for all sufficiently large $n$.

\newpage

\section{Problem 2 (40 pts): \texorpdfstring{$\NEXP \subseteq \coNEXP/_{n+1}$}{NEXP ⊆ coNEXP/{n+1}}}

\paragraph{Statement.}
Prove that $\NEXP \subseteq \coNEXP/_{n+1}$. In other words, for every language $L \in \NEXP$, prove that there exists an advice function $a(n) \in \{0,1\}^{n+1}$ such that there exists a coNEXP machine $M$ that, for every $n \in \mathbb{N}$, given the correct advice $a(n)$, $M$ computes $L$ on all $n$-bit inputs.

\newpage

\section{Problem 3 (40 pts): Prove \texorpdfstring{$\mathrm{CL} \subseteq \ZPP$}{CL ⊆ ZPP}}

The class $\mathrm{CL}$ (catalytic logspace) allows an algorithm logarithmic \emph{clean} space and polynomially many \emph{dirty} bits that must be restored at the end. Show that every $L \in \mathrm{CL}$ has a zero-error expected-polynomial-time algorithm.

\paragraph{Definition of ZPP.}
The complexity class $\ZPP$ (Zero-error Probabilistic Polynomial time) consists of all languages $L$ for which there exists a probabilistic polynomial-time Turing machine $M$ such that:
\begin{enumerate}
    \item For every input $x$, $M(x)$ outputs either $0$, $1$, or $\bot$ (``don't know'').
    \item If $x \in L$, then $\Pr[M(x) = 1] \geq \frac{1}{2}$ and $\Pr[M(x) = 0] = 0$.
    \item If $x \notin L$, then $\Pr[M(x) = 0] \geq \frac{1}{2}$ and $\Pr[M(x) = 1] = 0$.
    \item The expected running time of $M$ on any input is polynomial.
\end{enumerate}

Equivalently, $\ZPP$ is the class of languages that can be decided by Las Vegas algorithms: randomized algorithms that always give the correct answer when they terminate, and have polynomial expected running time.

\paragraph{Statement.}
Prove that $\mathrm{CL} \subseteq \ZPP$.


\newpage

\section{Problem 4 (40 pts): Prove that \texorpdfstring{$\TCone \subseteq \mathrm{CL}$}{TC$^1$ ⊆ CL}}

In the class we sketch the high-level idea of the proof that $\TCone \subseteq \mathrm{CL}$. In this problem, you will complete the proof.

\paragraph{Statement.}
Prove that $\TCone \subseteq \mathrm{CL}$. You should give a complete register program for the $\TCone$ circuit and prove it's correctness.

\newpage

\section*{Optional references and context}
For accessible background on catalytic logspace and the toggling construction, see:
\begin{itemize}
    \item Nathan Sheffield, \emph{A quick-and-dirty intro to CL}.
    \item Ian Mertz, \emph{Reusing Space: Techniques and Open Problems}.
\end{itemize}

\vfill
\noindent\textbf{(Optional) Solution placeholders}\\
You may use the following headings for your writeup; remove them if not needed.

\newpage
\section*{Solution to Problem 1}

\newpage
\section*{Solution to Problem 2}

\newpage
\section*{Solution to Problem 3}

\newpage
\section*{Solution to Problem 4}

\end{document}
